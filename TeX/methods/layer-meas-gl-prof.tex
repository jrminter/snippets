\subsection{Layer thickness measurement from gray level profiles}

Robust measurements of layer thickness requires both reliable
spatial calibration of the image and a robust method to locate
the image boundaries. Robust boundary detection is often more
difficult than one might anticipate. Boundaries are often rough
and different observers frequently choose different boundary
positions. The presence of noise in the image also complicates
this measurement.

One reproducible approach to boundary detection is to compute
the absolute value of the derivative of the gray level profile
across the boundary and to choose the maximum (representing the
greatest slope of the boundary) as the edge position. To reduce
the effect of noise, one can average many gray level profiles
which are recorded perpendicular to the boundary.

We have developed a custom tool chain that automates much
of this process. Think of this tool-chain as modules that may
be quickly configured to handle a wide range of problems.

In this instance, to provide an estimate
of the polymer layer thickness and to help understand the
image contrast, regions of the images of the FIB section
were identified that showed the best contrast at the boundary.
A line was drawn with the mouse along the projected interface
to determine the small tilt angle of the interface with respect
to the scan axis of the microscope. The image was rotated using
the analySIS Five software such that the interface was parallel
to the height of the image. A custom Imaging-C module was used
to compute the average gray-level profile across the image
and the data were stored as a comma-delimited text file (.csv).

The profile was analyzed using a series of custom scripts designed
for such measurements using the Open Source R statistical programming
language. The series was designed for a reproducible workflow.

The first script, 01-loadClean.R, loads the data from the
text file, computes the derivative using spline-smoothing
package `sfsmisc' developed by Martin Maechler \emph{et al.}
\footnote{Package ``sfsmisc,'' Version: 1.0-20, which provides
several useful utilities from Seminar fuer Statistik, ETH Zurich}
Our first script writes a binary data file containing the
position, mean gray level, and the spline-smoothed absolute
derivative ready for a peak search.

The second script, 02-measThickGm.R, reads the prepared
data frame, does a peak search (using some functions developed
by E. F. Glynn of the Stowers Institute for Medical Research.
The script finds the largest peaks (presumably the boundaries),
and computes the layer thickness. The script prepares an overlay
plot of the gray level profile (noisy), the spine-smoothed derivative
and the peak location. This version of the script simply finds
the peaks. We have extended this approach for cases where there
are well-defined approximately Gaussian peaks (electron diffraction
data) to fit the profile to a sum of Gaussian peaks to get better location
of the centroid. This preliminary analysis did not warrant that.
Finally, the layer thickness is stored to a R binary data file -- to
permit later automated processing to generate statistics from multiple
measurements.

\endinput
