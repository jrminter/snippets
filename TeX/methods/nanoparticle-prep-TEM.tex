\subsection{Specimen Preparation for TEM}

The sample was mixed in the vial using a vortex mixer. An
aliquot was removed and dispersed in distilled water to
produce a lightly colored dispersion in a 20 mL scintillation
vial. This dispersion was suspended using a vortex mixer.

A small aliquot (\til 15 \textmu L) was removed from the vial and
deposited across the light surface (grid side up) of a commercial
support (Ted Pella \#01824) which has a fragile, ultra-thin amorphous carbon
film supported by a lacy carbon net which is in turn supported by
a 400 mesh Cu TEM grid. Most of the droplet was blotted away with a
triangle of filter paper, leaving a fast-drying, thin, liquid film.

The particles imaged on the fragile ultra-thin amorphous carbon
film spanning the holes in the lacy net have the highest contrast
and resolution for TEM imaging and image processing.

\endinput
