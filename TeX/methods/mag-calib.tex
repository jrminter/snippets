\subsection{SEM Image Scale Calibration}

The image calibration factor [nm/pc] was measured for 15,000X
``XHD'' images recorded at a 5 mm working distance using the TLD.
The calibration artifact was the square array of holes with a 5
\textmu m spacing in the KMAG-\#96-19 primary calibration standard
manufactured for Kodak at Cornell. This standard was certified
by the National Physical Laboratory in Great
Britain\footnote{Certified 13-Jan-1998
with report number 08A031/9707/SEM4/121.}

Seven images of the square array of holes with a 5
\textmu m spacing were recorded under the same conditions
as the images of the nozzles, except using secondary electron
detection (this does not effect the scan of the microscope).

The images were analyzed using the ``KMag'' custom
Imaging-C module developed for this purpose. This module fits
the measured centroids from the holes to a rectangular lattice
that may be rotated with respect to the scan direction and permits
calibration along the scan direction (``X'') and the frame
direction (``Y''). These individual measurements were written
to a comma-delimited (.csv) text file.

A custom script (`01-calibMag.R') for the R\footnote{Available
at the Comprehensive R Archive Network \href{http://cran.r-project.org/}
{(CRAN).}} Open Source
statistical programming language automatically computed the
mean value of the calibration factors and their standard error
(95\% confidence interval). The
script also computed the aspect ratio for the scan (the pixels are
not square). The uncertainties in the two direction were added in
quadrature to estimate the standard error of the aspect ratio.
The script then generated the \LaTeX\ code for the results reported
in Table~\ref{tab:magCal}. This approach makes the results more
reproducible (they can be re-generated with a single command),
and eliminates errors arising from 'point, click, copy, and paste'
in Excel\footnote{The statistical community has long warned against
using Excel for statistical analysis. For example, see B.D. McCullough
and D. A. Heiser, `` On the accuracy of statistical procedures in
Microsoft Excel 2007,'' \emph{Computational Statistics and Data
Analysis}, \textbf{52}, 4570-4578 (2008).}.


These values were entered into the
control for the ``AnnularBseNozzle'' Imaging-C Module used to
analyze the backscattered electron images of the nozzles.


\endinput

